\documentclass[UTF8]{ctexart}
\usepackage[ruled, linesnumbered]{algorithm2e}
\usepackage{amsmath}
\usepackage{amsfonts}
\usepackage{amssymb}
\usepackage{amsthm}
\usepackage{boxedminipage}
\usepackage{circuitikz}
\usepackage{fancyhdr}
\usepackage{float}
\usepackage{fullpage}
\usepackage{gensymb}
\usepackage{geometry}
\usepackage{graphicx}
\usepackage{hyperref}
\usepackage{listings}
\usepackage{mflogo}
\usepackage{multirow}
\newcommand{\nop}[1]{}
\usepackage{siunitx}
\usepackage{subfigure}
\usepackage{texnames}
\usepackage{textcomp}
\usepackage{tikz}
\usepackage{titlesec}
\usepackage{wrapfig}
\usepackage{xcolor}
\usepackage{xltxtra}

\setlength{\headheight}{0pt}
\geometry{left=2.5cm,right=2.5cm,headsep=2cm,bottom=2cm,footnotesep=2cm}
\pagestyle{fancy}
\fancyfoot[C]{\thepage}
\fancyhead[L]{计算机科学与技术系}
\fancyhead[C]{
    \begin{huge}
        \textbf{}
    \end{huge}\\
    \vspace{0.4cm}
    \ \ \ \ \ \ \ \ }
\fancyhead[R]{\ \ \ \ \ \ \ \ \ \ \ \ 2023年10月7日}

\ctexset{
    section={
      format+ = \Large \raggedright,
     }
}

\newtheorem{theorem}{定理}
\newtheorem{lemma}{引理}
\newtheorem{definition}{定义}

\parindent0mm
%\pagestyle{empty}

\usetikzlibrary{arrows}

\newcounter{aufgc}
\newenvironment{exercise}[1]%
{\refstepcounter{aufgc}\textbf{问题 \arabic{aufgc}} \emph{#1} \\}
{\vspace{0.2cm}}%

\renewcommand{\labelenumi}{(\alph{enumi})}

\begin{document}

\textbf{Q1.}

\textit{解.}

(1) 正确
\begin{proof}
    因为 $\forall i\in\mathbb{N}^+,f_i(n)=O(g(n))$,则有
    $\exists c_i,n_i>0,\forall n>n_i,f_i(n)\leqslant c_{i}g(n)$,
    令 $n_0=\max\{n_1,n_2,\ldots\}$,
    有 $\forall n>n_0,f_i(n)\leqslant c_{i}g(n),F_m(n)=\sum_{i=1}^{m}f_i(n)\leqslant\sum_{i=1}^{m}c_{i}g(n)=\left(\sum_{i=1}^{m}c_{i}\right)g(n)$,
    故 $F_m(n)=O(g(n))$.
\end{proof}

(2) 错误,反例: $f_i(n)=2n,g(n)=n,f_i(n)=O(g(n)),F(n)=\sum_{i=1}^{n}f_i(n)=n(n-1)\neq O(g(n))$.

(3) 正确,可以取 $g(n)=n^2,f_1(n)=n,f_2(n)=n^{1.1},f_3(n)=n^{1.11},f_4(n)=n^{1.111},\ldots$

(4) 错误,反例: $f(n)=n(1-\sin(n)),g(n)=n(1+\sin(n)),f(n)+g(n)=2n=\Omega(n)$,
但 $f(n)$ 和 $g(n)$ 因为 $\forall k\in\mathbb{N}^+,f(2k\pi+\pi/2)=g(2k\pi+3\pi/2)=0$ 而
均不属于 $\Omega(n)$.

(5) 错误,反例: $f(n)=\begin{cases}n^{\alpha},\lfloor n\rfloor\mbox{是奇数}\\n^{\beta},\lfloor n\rfloor\mbox{是偶数}\end{cases}$,
显然 $f(n)=\Omega(n^{\alpha})$ 且 $f(n)=O(n^{\beta})$,但 $f(n+1)=\begin{cases}n^{\beta},\lfloor n\rfloor\mbox{是奇数}\\n^{\alpha},\lfloor n\rfloor\mbox{是偶数}\end{cases}$,
$f(n+1)\neq\Theta(f(n))$.

\hrulefill\medskip

\textbf{Q2.}

(1) \begin{proof}
    因为在三种情况中,总有 $A$ 的第一个参数不变大,而 $A$ 的第二个参数
    在第一个参数不变时总会变小,保证能变到 $0$ 使得第一个参数变小,而
    到第一个参数变小到 $0$ 时,递归就终止了,所以递归总能终止
\end{proof}

(2) \begin{proof}
    使用数学归纳法。

    首先当 $m=0$ 时,有 $A(m,n+1)=n+2>n+1=A(m,n)$ 对任意 $n$ 成立,也有
    $A(m+1,n)=A(m,A(m+1,n-1))=A(m,\ldots,A(m,A(m+1,0))\ldots)=A(m,\ldots,A(m,A(m,1))\ldots)=A(m,\ldots,A(m,2)\ldots)=n+2>n+1=A(m,n)$ 对任意 $n$ 成立

    假设当 $m\leqslant k$ 时,总有 $A(m,n+1)>A(m,n)$ 和 $A(m+1,n)>A(m,n)$ 对任意 $n$ 成立,

    因为 $m\leqslant k$ 时,$A(m+1,n)>A(m,n)>A(m-1,n)>\cdots A(0,n)$,

    对于 $m=k+1$,有 $A(m,n+1)=A(m-1,A(m,n))=A(k,A(m,n))>A(0,A(m,n))=A(m,n)+1>A(m,n)$;

    又因为 $m\leqslant k$ 时,

    则 $A(k+2,n-1)=A(k+1,\ldots,A(k+1,A(k+2,0))\ldots)=A(k+1,\ldots,A(k+1,A(k+1,1))\ldots)=A(k+1,\ldots,A(k+1,A(k,A(k+1,0)))\ldots)$
    中有 $n+1$ 层括号,每拆掉一层递归,至少要增加 $1$,故 $A(k+2,n-1)>n$

    由 $m=k+1$ 时 $A(m,n+1)>A(m,n)$,有$A(m+1,n)=A(k+2,n)=A(k+1,A(k+2,n-1))>A(k+1,n)=A(m,n)$
\end{proof}

(3)
\begin{theorem}
    $\alpha(x)=\omega(1)$
\end{theorem}
\begin{proof}
    $\forall c>0,\exists x_0=A(c+1,c+1),\forall x\geqslant x_0,0\leqslant c<c+1=\alpha(x_0)\leqslant \alpha(x)$,
    因此 $\alpha(x)=\omega(1)$
\end{proof}
\begin{theorem}
    $\alpha(x)=O(\lg^*x)$
\end{theorem}
\begin{proof}
    可以把 $A(m,n)$ 看作一个矩阵,其中第一行 $A(0,n)=n+1$,$A(m,0)=A(m-1,1)$,
    \[A(m,n)=\underbrace{A(m-1,A(m-1,\ldots,A(m-1,}_{n\mbox{\zihao{7}层}}A(m,0))))=\underbrace{A(m-1,A(m-1,\ldots,A(m-1,}_{n\mbox{\zihao{7}层}}A(m-1,1))))\]
    因此,有 $A(1,n)=(\ldots(A(0,1)+1)+\ldots)+1=n+2$,$A(2,n)=(\ldots(A(1,1)+2)+\ldots)+2=2n+3$,$A(3,n)=2(\ldots2(2A(2,1)+3)+\ldots)+3=2^{n+3}-3$,$A(4,n)=2^{\ldots^{2^{A(3,1)-3}-3}-3}-3$,
    而 $\lg^*(n)$ 相当于求一个指数塔 $2^{2^{\cdots^{2}}}$ 的层数,
    显然,$A(4,n)$ 已经和指数塔量级相同,$A(5,n)$ 会远大于它,
    即 $n>4$ 时,\[A(n,n)\gg A(4,n)>\left.2^{2^{\cdots^2}}\right\}n\mbox{层}\]
    因此 \[\lg^{(n)}(A(n,n))>\lg^{(n)}\left(\left.2^{2^{\cdots^2}}\right\}n\mbox{层}\right)=1\]
    取 $n=\alpha(x)$,有 $1<\lg^{(\alpha(x))}(A(\alpha(x),A(\alpha(x))))\leqslant \lg^{(\alpha(x))}(x)$,
    故 $0\leqslant\alpha(x)<\lg^*(x)$,因此 $\alpha(x)=O(\lg^*(x))$.
\end{proof}

\hrulefill\medskip

\textbf{Q3.}

(1) \textit{解.}

1. $n^2=\Omega(n^{\log_{2}3+\varepsilon})$,且 $3\cdot {(n/2)}^2=3n^2/4\leqslant 7n^2/8$,故 $T(n)=\Theta(n^2)$

2. $n^2=\Theta(n^{\log_{2}4})$,故 $T(n)=\Theta(n^2\lg n)$

3. $2^n=\Omega(n^{\log_{2}1+\varepsilon})$,且 $2^{n/2}\leqslant 2^n/2$,故 $T(n)=\Theta(2^n)$

4. $n^n=\Theta(n^{\log_{2}2^n})$,故 $T(n)=\Theta(n^n\lg n)$

5. $n=O(n^{\log_{4}16-\varepsilon})$,故 $T(n)=\Theta(n^2)$

6. $n\lg n=\Omega(n^{\log_{2}2}+\varepsilon)$,但不存在 $c$ 使得 $n$ 足够大时,有 $2n/2\cdot\lg(n/2)\leqslant cn\lg n$,因为 $\forall c<1$,取 $n>\frac{1}{1-c}$ 即可使得 $n\lg(n/2)>cn\lg n$,故不适用主方法

7. $\frac{n}{\lg n}$ 无法与 $O(n^{\log_{2}2-\varepsilon})$ 或 $\Theta(n^{\log_{2}2})$ 比较,不适用主方法.

8. $n^{0.51}=\Omega(n^{\log_{4}2}+\varepsilon)$,且 $2\cdot {(n/4)}^{0.51}\leqslant0.99n^{0.51}$,故 $T(n)=\Theta(n^{0.51})$

9. $0.5<1$,故不适用主方法

10. $n!=\Omega(n^{\log_{4}16+\varepsilon})$,且 $n>8$ 时 $16(n/4)!\leqslant 0.5n!$,故 $T(n)=\Theta(n!)$

11. $\lg n=O(n^{\log_{2}\sqrt{2}-\varepsilon})$,故 $T(n)=\Theta(n^{0.5})$

12. $n=O(n^{\log_{2}3-\varepsilon})$,故 $T(n)=\Theta(n^{\log_{3}2})$

13. $\sqrt{n}=O(n^{\log_{3}3-\varepsilon})$,故 $T(n)=\Theta(n)$

14. $cn=O(n^{\log_{2}4-\varepsilon})$,故 $T(n)=\Theta(n^2)$

15. $n\lg n=\Omega(n^{\log_{4}3+\varepsilon})$,且 $3\cdot \frac{n}{4}\lg \frac{n}{4}\leqslant \frac{7}{8}n\lg n$,故 $T(n)=\Theta(n\lg n)$

16. $\frac{n}{2}=\Theta(n^{\log_{3}3})$,故 $T(n)=\Theta(n\lg n)$

17. $n^2\lg n=\Omega(n^{\log_{3}6+\varepsilon})$,且 $6\cdot{(n/3)}^2\lg (n/3)\leqslant\frac{5}{6}n^2\lg n$,故 $T(n)=\Theta(n^2\lg n)$

18. $\frac{n}{\lg n}=O(n^{\log_{2}4-\varepsilon})$,故 $T(n)=\Theta(n^2)$

19. $-n^2\lg n$ 不是正函数,故不适用主方法

20. $n^2=\Omega(n^{\log_{3}7+\varepsilon})$,且 $7{(n/3)}^2\leqslant\frac{8}{9}n^2$,故 $T(n)=\Theta(n^{\log_{3}7})$

21. $\lg n=O(n^{\log_{2}4-\varepsilon})$,故 $T(n)=\Theta(n^2)$

22. $n(2-\cos n)=\Omega(n^{\log_{2}1+\varepsilon})$,但 $n=(2k+1)\pi$ 时 $(n/2)(2-\cos(n/2))=\frac{3n}{2}>cn=cn(2-\cos n)$,故不适用主方法

23. $n\lg n\lg\lg n=\Omega(n^{\log_{2}2+\varepsilon})$,但 $n$ 足够大时,有 $n\lg (n/2)\lg\lg(n/2)>cn\lg n\lg\lg n$,故不适用主方法

24. $n{(\lg\lg n)}^r=\Omega(n^{\log_{2}2+\varepsilon})$,且 $n$ 足够大时,但因为 $r$ 可正可负,无法比较 $2\cdot(n/2){(\lg\lg(n/2))}^r$ 与 $cn{(\lg\lg n)}^r$ 的大小,故不适用主方法

25. $n\frac{{(\lg\lg\lg n)}^2}{\lg n\lg\lg n}$ 无法与 $O(n^{\log_{2}2-\varepsilon})$ 比较,故不适用主方法

(2)

7. $\alpha=\log_{2}2=1$,$EL^{\bf{e}}(n)=n\cdot{(\lg n)}^{-1}$,即 $\bf{e}=(1,-1)$,${\rm cord}(\bf{e})=2,{\rm cpow}(\bf{e})=0$,则有 $e_0=\alpha$ 且 ${\rm cpow}(\bf{e})>-1$,故 $T(n)=\Theta(\frac{n}{\lg n}\lg n\lg \lg n)=n\lg\lg n$

23. $\alpha=\log_{2}2=1$,$EL^{\bf{e}}(n)=n\cdot\lg n\cdot\lg\lg n$,即$\bf{e}=(1,1,1)$,${\rm cord}(\bf{e})=1,{\rm cpow}(\bf{e})=1$,则有 $e_0=\alpha$ 且 ${\rm cpow}(\bf{e})>-1$,故 $T(n)=\Theta(n\lg n\lg\lg n\lg n)=\Theta(n{(\lg n)}^2(\lg\lg n))$

24. $\alpha=\log_{2}2=1$,$EL^{\bf{e}}(n)=n{(\lg\lg n)}^r$,即 $\bf{e}=(1,0,r)$,${\rm cord}(\bf{e})=1,{\rm cpow}(\bf{e})=0$,则有 $e_0=\alpha$ 且 ${\rm cpow}(\bf{e})>-1$,故 $T(n)=\Theta(n{(\lg\lg n)}^r\lg n)=\Theta(n\lg n{(\lg\lg n)}^r)$

25. $\alpha=\log_{2}2=1$,$EL^{\bf{e}}(n)=n\frac{{(\lg\lg\lg n)}^s}{\lg n\lg\lg n}$,即 $\bf{e}=(1,-1,-1,s)$,${\rm cord}(\bf{e})=3,{\rm cpow}(\bf{e})=s$,则若 $s>0$,有 $e_0=\alpha$ 且 ${\rm cpow}(\bf{e})>-1$,故 $T(n)=\Theta(n\frac{{(\lg\lg\lg n)}^s}{\lg n\lg\lg n}\lg n\lg\lg n\lg\lg\lg n)=\Theta(n{(\lg\lg\lg n)}^{s+1})$,若 $s\leqslant-1$,则 $T(n)=\Theta(n)$

\hrulefill\medskip

\textbf{Q4.}

(1) \textit{解.}

1. $a=\frac{|f-\frac{2n}{f}|+1}{2},b=\frac{f+\frac{2n}{f}-1}{2}$

2. 如果 $a+b$ 为奇数,则 $f=a+b$,否则 $f=b-a+1$

(2) \textit{解.} 正确性:将 $n$ 质因数分解,有 $n=2^{t_0}p_1^{t_1}\cdots p_s^{t_s},t_0,t_1,\ldots,t_s>0$,
第一次循环结束的 $n=p_1^{t_1}\cdots p_s^{t_s}$,只含有奇质因子,
接下来的循环每次选取不大于 $\sqrt{n}$ 的一个质因子,得到 $S=(t_1+1)(t_2+1)\cdots(t_r+1)$,
如果循环到 $n=p_{s-1}p_s$,显然 $p_{s-1}^2<n$,循环不会终止,
只有在 $p_s^2>n$ 时有最终的 $n\neq1$,
故循环终止时有此时的 $n=1$ 或者 $n=p_s$.

因此,若 $n=2^{t_0}p_1^{t_1}\cdots p_s$,$S=(t_1+1)(t_2+1)\cdots(t_{s-1}+1)\cdot 2$;
若 $n=2^{t_0}p_1^{t_1}\cdots p_s^{t_s},t_s>1$,$S=(t_1+1)(t_2+1)\cdots(t_s+1)$.
其实也就是 $S=(t_1+1)(t_2+1)\cdots(t_s+1)$,而这也是所有 $n$ 的奇因子数,
由第 1 问知每一个 $n$ 的奇因子和 $S_n$ 中的一个 $\langle a,b\rangle$ 一一对应,所以
输出的 $S$ 就是 $|S_n|$.

\begin{theorem}
    算法1的最坏时间复杂度为 $\Theta(2^{\frac{m}{2}})$
\end{theorem}

\begin{proof}
    最坏时有输入的 $n$ 为一个 $m$ 位的二进制质数,则循环直到 $p>\sqrt{n}$
    时才停下,而每一次循环 $p+1$,因此此时的时间复杂度为 $\Theta(\sqrt{2^m})=\Theta(2^{\frac{m}{2}})$

    接下来只需证明 $n$ 不是质数时,总能使时间复杂度不高于 $\Theta(2^{\frac{m}{2}})$

    当 $n$ 不是质数时,有 $n=p_1^{t_1}\cdots p_s^{t_s},t_1,\ldots,t_s>0$,
    则总有质因子 $p_i\leqslant\sqrt{n}$,否则取最小的质因子 $p_i>\sqrt{n}$,
    $p_i^2>n$,则不存在 $p_{i+1}$,不然 $p_i^2<p_{i}p_{i+1}\leqslant n$矛盾,
    也就是 $p_i$ 也是最大的质因子,此时有 $n$ 为质数,矛盾,因此总能有 $p_i$
    满足循环,使得复杂度不高于 $\Theta(2^{\frac{m}{2}})$.
\end{proof}

(3.1) 可以认为最坏时间复杂度是 $\Omega(2^{\alpha m})\cap O(2^{\beta m})$,
但不能认为平均时间复杂度为 $\Theta(2^{\gamma m})$,因为这需要看 $L$ 的分布.

(3.2) 绘图如下,最大值点为 0.33,最大值为 5.13.

\begin{figure}[htbp]
    \begin{center}
        \includegraphics[width=0.8\textwidth]{Figure.png}    
    \end{center}
\end{figure}

\end{document}